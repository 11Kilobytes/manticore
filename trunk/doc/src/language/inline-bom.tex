\documentclass[11pt]{article}

% common LaTeX macros
%
% Last modified: 03-02-2007
%

\usepackage{times}
%-------------------------
% the following magic makes the tt font in math mode be the same as the
% normal tt font (i.e., Courier)
%
\SetMathAlphabet{\mathtt}{normal}{OT1}{pcr}{n}{n}
\SetMathAlphabet{\mathtt}{bold}{OT1}{pcr}{bx}{n}
%-------------------------

\usepackage{amsmath}
\usepackage{amssymb} % for \pitchfork

\newcommand{\NOTE}[1]{%
  \par\leavevmode\noindent\textbf{[[ #1 ]]}\par\leavevmode\noindent}
\newcommand{\CUT}[1]{}
\newcommand{\SIDENOTE}[1]{%
  \marginpar{\tiny\raggedright{#1}}}

\newcommand{\appref}[1]{Appendix~\ref{#1}}
\newcommand{\chapref}[1]{Chapter~\ref{#1}}
\newcommand{\secref}[1]{Section~\ref{#1}}
\newcommand{\tblref}[1]{Table~\ref{#1}}
\newcommand{\figref}[1]{Figure~\ref{#1}}
\newcommand{\listingref}[1]{Listing~\ref{#1}}
\newcommand{\pref}[1]{{page~\pageref{#1}}}
\newcommand{\defref}[1]{Definition~\ref{#1}}
\newcommand{\ruleref}[1]{Rule~\ref{#1}}

\newcommand{\eg}{{\em e.g.}}
\newcommand{\cf}{{\em cf.}}
\newcommand{\ie}{{\em i.e.}}
\newcommand{\etc}{{\em etc.\/}}
\newcommand{\naive}{na\"{\i}ve}
\newcommand{\ala}{{\em \`{a} la\/}}
\newcommand{\etal}{{\em et al.\/}}
\newcommand{\role}{r\^{o}le}
\newcommand{\vs}{{\em vs.}}
\newcommand{\forte}{{fort\'{e}\/}}

%
% language names
\newcommand{\Cplusplus}{\mbox{C\hspace{-.05em}\raisebox{.4ex}{\tiny\bf ++}}}
\newcommand{\Cmm}{\mbox{C\hspace{-.05em}\raisebox{.4ex}{\small\textbf{{-}{-}}}}}
\newcommand{\csharp}{\textsc{C\#}}
\newcommand{\C}{\textsc{C}}
\newcommand{\Ckit}{\textsc{Ckit}}
\newcommand{\java}{\textsc{Java}}
\newcommand{\loom}{\textsc{Loom}}
\newcommand{\moby}{\textsc{Moby}}
\newcommand{\minimoby}{\textsc{MiniMoby}}
\newcommand{\micromoby}{\textsc{microMoby}}
\newcommand{\MOC}{\textsc{MOC}}
\newcommand{\ml}{\textsc{ML}}
\newcommand{\sml}{\textsc{SML}}
\newcommand{\smlnj}{\textsc{SML/NJ}}
\newcommand{\mlj}{\textsc{MLj}}
\newcommand{\cml}{\textsc{CML}}
\newcommand{\pml}{\textsc{PML}}
\newcommand{\ocaml}{\textsc{OCaml}}
\newcommand{\mlkk}{\textsc{ML2000}}
\newcommand{\haskell}{\textsc{Haskell}}
\newcommand{\mltwok}{\textsc{ML2000}}
\newcommand{\scala}{\textsc{Scala}}
\newcommand{\perl}{\textsc{Perl}}
\newcommand{\scheme}{\textsc{Scheme}}
\newcommand{\unix}{\textsc{Unix}}
\newcommand{\smalltalk}{\textsc{Smalltalk}}
\newcommand{\self}{\textsc{Self}}

%
% font commands
\providecommand{\bftt}[1]{{\ttfamily\bfseries{}#1}}
\providecommand{\ittt}[1]{{\ttfamily\itshape{}#1}}
\providecommand{\kw}[1]{\bftt{#1}}
\providecommand{\nt}[1]{{\rmfamily\itshape{#1}}}
\providecommand{\term}[1]{{\sffamily{#1}}}
%
% math-mode versions
\providecommand{\mkw}[1]{\ensuremath{\text{\kw{#1}}}}
\providecommand{\mnt}[1]{\ensuremath{\text{\nt{#1}}}}
\providecommand{\mterm}[1]{\ensuremath{\text{\term{#1}}}}

% braces (in math mode)
\newcommand{\LCB}{\mkw{\{}}
\newcommand{\RCB}{\mkw{\}}}

% underscore
\newcommand{\US}{\char`\_}

%%%%%
% Some common math notation
%

% double brackets
\newcommand{\LDB}{\ensuremath{[\mskip -3mu [}}
\newcommand{\RDB}{\ensuremath{]\mskip -3mu ]}}

\newcommand{\dom}{\ensuremath{\mathrm{dom}}}
\newcommand{\rng}{\ensuremath{\mathrm{rng}}}

% sets
\newcommand{\SET}[1]{\ensuremath{\{#1\}}}
\newcommand{\Fin}{\textrm{Fin}}     % finite power set
\newcommand{\DISJOINT}[2]{\ensuremath{#1 \pitchfork #2}}
\newcommand{\finsubset}{\mathrel{\stackrel{\textrm{fin}}{\subset}}}


% finite maps
\newcommand{\finmap}{\mathrel{\stackrel{\textrm{fin}}{\rightarrow}}}
\newcommand{\MAP}[2]{\SET{#1 \mapsto #2}}
\newcommand{\EXTEND}[2]{\ensuremath{#1{\pm}#2}}
\newcommand{\EXTENDone}[3]{\EXTEND{#1}{\MAP{#2}{#3}}}
\newcommand{\SUBST}[3]{\ensuremath{#1[#2\mapsto{}#3]}}
\newcommand{\SUBSTTWO}[5]{\ensuremath{#1[#2\mapsto{}#3,#4\mapsto{}#5]}}


% timestamp
\newcount\timeHH
\newcount\timeMM
\timeHH=\time
\divide\timeHH by 60
\timeMM=\time
\count255=\timeHH
\multiply\count255 by -60 \advance\timeMM by \count255
\newcommand{\timestamp}{%
  \today{} ---
  \ifnum\timeHH<10 0\fi\number\timeHH\,:\,\ifnum\timeMM<10 0\fi\number\timeMM}


\usepackage{graphicx}
\usepackage{../common/code}

\title{Inline BOM}
\author{The Manticore Group}
\date{Draft of \today}

\begin{document}
\maketitle

\section{Overview}
Inline BOM is a language extension that allows us to embed BOM code in PML programs. It is the home of our \emph{scheduling language}, in which we build the basis for our high-level parallel constructs\cite{manticore-sched-icfp08}. We also use inline BOM to implement a variety of other language mechanisms, including some of the following.
\begin{itemize}
\item process schedulers
\item parallel constructs, \eg{}, futures, \texttt{pval}, \etc{}
\item primitive operations, \eg{}, operations over 32-bit ints
\item foreign calls
\item synchronization / concurrent data structures, \ie{}, implementing CML
\end{itemize}

\subsection{Rationale}
One might reasonably as why we use inline BOM. Our primary motivation is to separate unsafe language features out of PML. Doing so gives PML a cleaner semantics, thereby simplifying the development model and enablinig more aggressive optimizations. We can use inline BOM as a sandbox for our unsafe code.

Inline BOM includes the following list of features not found in PML.
\begin{itemize}
\item scheduling primitives (\texttt{run}, \texttt{forward}, \etc{})
\item atomic operations, \eg{} compare-and-swap
\item precise byte layouts of objects
\item mutable memory
\item control over whether an object gets allocated in the local or global heap
\item C calls
\item primitive operations
\end{itemize}

\section{Design}
We have designed inline BOM as a conservative extension of our module language. Below we describe these forms and their semantics.

\subsection{Inline BOM declarations}
Primcode declarations allow us to make BOM declarations, which are types, C function prototypes, or HLOps definitions. Primcode declarations occur in top-level declarations of modules. They obey the same scoping rules as other top-level declarations.
\begin{centercode}
  _primcode (
    BOMDecl ...
  )
\end{centercode}

\subsubsection{Example: BOM type declaration}
We can define the state field of single-toucher futures as follows. The state can be either a non-pointer flag, or a pointer to either the value of the future or the continuation of a blocked thread. 
\begin{centercode}
  _primcode( 
    (*
     * a future_state word contains one of the following values:
     *          EMPTY_F
     *          STOLEN_F
     *          EVAL_F
     *          FULL      value
     *          WAITING   cont
     *)
    typedef future1_state = any; 
  )
\end{centercode}

\subsubsection{Example: C function prototype}
The following snippet contains a prototype for a print function, which is defined in some external C file.
\begin{centercode}
  _primcode(
    extern void M_Print(void*)
  )
\end{centercode}

\subsubsection{Example: HLOps}
The following HLOp prints its parameter string.
\begin{centercode}
  _primcode(
    define inline @print-ln(msg : string / exh : exh) : () =
      do ccall M_Print(msg)
      return ()
  )
\end{centercode}

\subsection{Types}
We can make type declarations that are visible to PML and BOM code. In PML, the type t is an abstract type, but in BOM, it has the type bomTy.
\begin{centercode}
  type ('a, ...) t = _prim(bomTy)
\end{centercode}

\subsubsection{Example}
We can define the type of single-toucher futures as follows. The future structure is just a mutable pair of the state and the thunk.
\begin{centercode}
  type future1 = _prim(![future1_state, thunk])
\end{centercode}

\subsection{Exporting BOM definitions}
We can bind BOM functions and HLOps to PML identifiers. Because the translation from BOM types to PML types is undefined, we require the programmer to ascribe the bound variable a PML type. 
\begin{centercode}
  val f : ty = _prim(bomId)
\end{centercode}

\subsubsection{Example}
We export the HLOp \texttt{@f} to PML as the function \texttt{f}.
\begin{centercode}
  _primcode(
    define @f(_ : unit / exh : exh) : unit = ...;
  )
  val f : unit -> unit = _prim(@f)
\end{centercode}

\subsection{Importing PML datatypes}
BOM code can refer to datatypes and constructors defined in PML.

\subsubsection{Example}
The HLOp \texttt{mk-x} creates an element of the datatype \texttt{t}.
\begin{centercode}
  datatype t = X of int | Y
  
  _primcode(
    define @mk-x(x : int / exh : exh) : t =
      return(X(x))
  )
\end{centercode}

\subsection{Importing PML declarations}
BOM code can refer to PML variables that are bound at the type level.
\begin{centercode}
  let f : bomTy = pmlvar(pmlId)
\end{centercode}

\subsubsection{Example}
Here we bind the list application function for use in BOM.
\begin{centercode}
  let app : fun( [fun(any / exh -> unit), List.list] / exh) = 
                     pmlvar(List.app)
\end{centercode}

\bibliographystyle{alpha}
\bibliography{../common/manticore}

\end{document}
