\documentclass[11pt]{article}

% common LaTeX macros
%
% Last modified: 03-02-2007
%

\usepackage{times}
%-------------------------
% the following magic makes the tt font in math mode be the same as the
% normal tt font (i.e., Courier)
%
\SetMathAlphabet{\mathtt}{normal}{OT1}{pcr}{n}{n}
\SetMathAlphabet{\mathtt}{bold}{OT1}{pcr}{bx}{n}
%-------------------------

\usepackage{amsmath}
\usepackage{amssymb} % for \pitchfork

\newcommand{\NOTE}[1]{%
  \par\leavevmode\noindent\textbf{[[ #1 ]]}\par\leavevmode\noindent}
\newcommand{\CUT}[1]{}
\newcommand{\SIDENOTE}[1]{%
  \marginpar{\tiny\raggedright{#1}}}

\newcommand{\appref}[1]{Appendix~\ref{#1}}
\newcommand{\chapref}[1]{Chapter~\ref{#1}}
\newcommand{\secref}[1]{Section~\ref{#1}}
\newcommand{\tblref}[1]{Table~\ref{#1}}
\newcommand{\figref}[1]{Figure~\ref{#1}}
\newcommand{\listingref}[1]{Listing~\ref{#1}}
\newcommand{\pref}[1]{{page~\pageref{#1}}}
\newcommand{\defref}[1]{Definition~\ref{#1}}
\newcommand{\ruleref}[1]{Rule~\ref{#1}}

\newcommand{\eg}{{\em e.g.}}
\newcommand{\cf}{{\em cf.}}
\newcommand{\ie}{{\em i.e.}}
\newcommand{\etc}{{\em etc.\/}}
\newcommand{\naive}{na\"{\i}ve}
\newcommand{\ala}{{\em \`{a} la\/}}
\newcommand{\etal}{{\em et al.\/}}
\newcommand{\role}{r\^{o}le}
\newcommand{\vs}{{\em vs.}}
\newcommand{\forte}{{fort\'{e}\/}}

%
% language names
\newcommand{\Cplusplus}{\mbox{C\hspace{-.05em}\raisebox{.4ex}{\tiny\bf ++}}}
\newcommand{\Cmm}{\mbox{C\hspace{-.05em}\raisebox{.4ex}{\small\textbf{{-}{-}}}}}
\newcommand{\csharp}{\textsc{C\#}}
\newcommand{\C}{\textsc{C}}
\newcommand{\Ckit}{\textsc{Ckit}}
\newcommand{\java}{\textsc{Java}}
\newcommand{\loom}{\textsc{Loom}}
\newcommand{\moby}{\textsc{Moby}}
\newcommand{\minimoby}{\textsc{MiniMoby}}
\newcommand{\micromoby}{\textsc{microMoby}}
\newcommand{\MOC}{\textsc{MOC}}
\newcommand{\ml}{\textsc{ML}}
\newcommand{\sml}{\textsc{SML}}
\newcommand{\smlnj}{\textsc{SML/NJ}}
\newcommand{\mlj}{\textsc{MLj}}
\newcommand{\cml}{\textsc{CML}}
\newcommand{\pml}{\textsc{PML}}
\newcommand{\ocaml}{\textsc{OCaml}}
\newcommand{\mlkk}{\textsc{ML2000}}
\newcommand{\haskell}{\textsc{Haskell}}
\newcommand{\mltwok}{\textsc{ML2000}}
\newcommand{\scala}{\textsc{Scala}}
\newcommand{\perl}{\textsc{Perl}}
\newcommand{\scheme}{\textsc{Scheme}}
\newcommand{\unix}{\textsc{Unix}}
\newcommand{\smalltalk}{\textsc{Smalltalk}}
\newcommand{\self}{\textsc{Self}}

%
% font commands
\providecommand{\bftt}[1]{{\ttfamily\bfseries{}#1}}
\providecommand{\ittt}[1]{{\ttfamily\itshape{}#1}}
\providecommand{\kw}[1]{\bftt{#1}}
\providecommand{\nt}[1]{{\rmfamily\itshape{#1}}}
\providecommand{\term}[1]{{\sffamily{#1}}}
%
% math-mode versions
\providecommand{\mkw}[1]{\ensuremath{\text{\kw{#1}}}}
\providecommand{\mnt}[1]{\ensuremath{\text{\nt{#1}}}}
\providecommand{\mterm}[1]{\ensuremath{\text{\term{#1}}}}

% braces (in math mode)
\newcommand{\LCB}{\mkw{\{}}
\newcommand{\RCB}{\mkw{\}}}

% underscore
\newcommand{\US}{\char`\_}

%%%%%
% Some common math notation
%

% double brackets
\newcommand{\LDB}{\ensuremath{[\mskip -3mu [}}
\newcommand{\RDB}{\ensuremath{]\mskip -3mu ]}}

\newcommand{\dom}{\ensuremath{\mathrm{dom}}}
\newcommand{\rng}{\ensuremath{\mathrm{rng}}}

% sets
\newcommand{\SET}[1]{\ensuremath{\{#1\}}}
\newcommand{\Fin}{\textrm{Fin}}     % finite power set
\newcommand{\DISJOINT}[2]{\ensuremath{#1 \pitchfork #2}}
\newcommand{\finsubset}{\mathrel{\stackrel{\textrm{fin}}{\subset}}}


% finite maps
\newcommand{\finmap}{\mathrel{\stackrel{\textrm{fin}}{\rightarrow}}}
\newcommand{\MAP}[2]{\SET{#1 \mapsto #2}}
\newcommand{\EXTEND}[2]{\ensuremath{#1{\pm}#2}}
\newcommand{\EXTENDone}[3]{\EXTEND{#1}{\MAP{#2}{#3}}}
\newcommand{\SUBST}[3]{\ensuremath{#1[#2\mapsto{}#3]}}
\newcommand{\SUBSTTWO}[5]{\ensuremath{#1[#2\mapsto{}#3,#4\mapsto{}#5]}}


% timestamp
\newcount\timeHH
\newcount\timeMM
\timeHH=\time
\divide\timeHH by 60
\timeMM=\time
\count255=\timeHH
\multiply\count255 by -60 \advance\timeMM by \count255
\newcommand{\timestamp}{%
  \today{} ---
  \ifnum\timeHH<10 0\fi\number\timeHH\,:\,\ifnum\timeMM<10 0\fi\number\timeMM}


\usepackage{graphicx}
\usepackage{../common/code}

\title{Manticore Implementation Note \\ Fiber Local Storage}
\author{The Manticore Group}
\date{Draft of \today}

\begin{document}
\maketitle

Fiber-local storage (fls) is a mechanism for associating a fiber with some implicit, local state. There are three core operations. The first creates an empty local storage; the second sets the fls for the host vproc; the third gets the fls on the host vproc.
\begin{centercode}
  type fls

  val new : unit -> fls
  val set : fls -> unit
  val get : unit -> fls
\end{centercode}

To make fls useful, we provide some basic dictionary operations. Keys into the dictionary are represented by the \texttt{tag} type. The tag type is a phantom type, i.e.,
\begin{centercode}
  type 'a tag = int
  val key1 : key1_ty tag = 1
  val key2 : key2_ty tag = 2
  ...
\end{centercode}
We have operations for adding and finding elements.
\begin{centercode}
  val add : (fls * ('a tag * 'a)) -> fls
  val find : (fls * 'a tag) -> 'a option
\end{centercode}

\section{Fiber-group storage}
In certain cases, we wish to associate some implicit, shared state with a \emph{group} of fibers. For read-only access, our fls mechanism is already sufficient. But for read-write access, we need synchronized memory.

\subsection{Example: scheduler initialization for futures}
Our architecture allows each thread to use multiple schedulers. To avoid unecessary overhead, we delay initialization until the last possible moment, when the first spawn operation occurs. 

Unfortunately, because of our architecture, we need to synchronize the initialization of schedulers. Suppose that we have two alternative implementations of futures, \texttt{F1} and \texttt{F2}. It is easy to imagine a program, such as the one below, that uses both schedulers. Unfortunately, we cannot know which call to \texttt{F2.future} happens first, so we need some synchronized memory.
\begin{centercode}
structure F1 : FUTURE = ...
structure F2 : FUTURE = ...

fun f () = let
    val fut2 = F2.future(fn () => e3)
    in
       e2 + F2.touch fut2
    end

val fut1 = F1.future(fn () => e1 + f())
val fut2 = F1.future(fn () => e1 + f())

val x = if foo()
	   then F1.touch fut1 + F1.touch fut2
	else ( F1.cancel fut1; F1.touch fut1)
\end{centercode}

The code below handles the initialization of the scheduler for \texttt{F1}. The code operates as follows. We first look for shared scheduler state in fls. Using set-once memory, we get a handle on the ready queue. If the scheduler has already initialized, we simple read the pointer from the set-once memory. If not, we initialize the scheduler while holding a lock. This lock prevents other fibers from initializing the scheduler independently.
%\input{example-code/future1-get-ready-queue}

Now it easy easy to encode the spawn operations for \texttt{F1} futures.
%\input{example-code/future1-future}

\section{Thread capabilities}
Because threads consist of one or more fibers, they necessarily have read-only access to fls. Thus, any local storage entry must be added during the \texttt{spawn} operation. To address this issue, we use the notion of \emph{thread capabilities}. For example, a given thread could have a capability for depth-first or breadth-first futures.
\begin{centercode}
  spawn(f1, FutureDFS.capability);
  spawn(f2, FutureBFS.capability)
\end{centercode}

Capabilities are just entries in the fiber-local storage dictionary. Before spawning a CML thread, we add all the capabilities to its local storage.
\begin{centercode}
  type 'a capability = 'a tag * 'a
  fun spawn (thunk, capabilities) = let
        val fls = List.foldl FLS.add (FLS.new()) capabilities
        in
            VProcQueue.enqueue(fls, Control.fiber thunk);
            fls
        end
\end{centercode}

Our implementation defines a set of default capabilities for threads. See the file
\begin{center}
 \texttt{src/lib/basis/runtime/utils/default-thread-capabilities.pml} 
\end{center}
for the complete list.

\end{document}
