\documentclass[11pt]{article}

\usepackage{amscd}
\usepackage{amsmath}
\usepackage[pdftex]{graphicx}
\usepackage{ifthen}
\usepackage{url}

\usepackage{../common/code}
% common LaTeX macros
%
% Last modified: 03-02-2007
%

\usepackage{times}
%-------------------------
% the following magic makes the tt font in math mode be the same as the
% normal tt font (i.e., Courier)
%
\SetMathAlphabet{\mathtt}{normal}{OT1}{pcr}{n}{n}
\SetMathAlphabet{\mathtt}{bold}{OT1}{pcr}{bx}{n}
%-------------------------

\usepackage{amsmath}
\usepackage{amssymb} % for \pitchfork

\newcommand{\NOTE}[1]{%
  \par\leavevmode\noindent\textbf{[[ #1 ]]}\par\leavevmode\noindent}
\newcommand{\CUT}[1]{}
\newcommand{\SIDENOTE}[1]{%
  \marginpar{\tiny\raggedright{#1}}}

\newcommand{\appref}[1]{Appendix~\ref{#1}}
\newcommand{\chapref}[1]{Chapter~\ref{#1}}
\newcommand{\secref}[1]{Section~\ref{#1}}
\newcommand{\tblref}[1]{Table~\ref{#1}}
\newcommand{\figref}[1]{Figure~\ref{#1}}
\newcommand{\listingref}[1]{Listing~\ref{#1}}
\newcommand{\pref}[1]{{page~\pageref{#1}}}
\newcommand{\defref}[1]{Definition~\ref{#1}}
\newcommand{\ruleref}[1]{Rule~\ref{#1}}

\newcommand{\eg}{{\em e.g.}}
\newcommand{\cf}{{\em cf.}}
\newcommand{\ie}{{\em i.e.}}
\newcommand{\etc}{{\em etc.\/}}
\newcommand{\naive}{na\"{\i}ve}
\newcommand{\ala}{{\em \`{a} la\/}}
\newcommand{\etal}{{\em et al.\/}}
\newcommand{\role}{r\^{o}le}
\newcommand{\vs}{{\em vs.}}
\newcommand{\forte}{{fort\'{e}\/}}

%
% language names
\newcommand{\Cplusplus}{\mbox{C\hspace{-.05em}\raisebox{.4ex}{\tiny\bf ++}}}
\newcommand{\Cmm}{\mbox{C\hspace{-.05em}\raisebox{.4ex}{\small\textbf{{-}{-}}}}}
\newcommand{\csharp}{\textsc{C\#}}
\newcommand{\C}{\textsc{C}}
\newcommand{\Ckit}{\textsc{Ckit}}
\newcommand{\java}{\textsc{Java}}
\newcommand{\loom}{\textsc{Loom}}
\newcommand{\moby}{\textsc{Moby}}
\newcommand{\minimoby}{\textsc{MiniMoby}}
\newcommand{\micromoby}{\textsc{microMoby}}
\newcommand{\MOC}{\textsc{MOC}}
\newcommand{\ml}{\textsc{ML}}
\newcommand{\sml}{\textsc{SML}}
\newcommand{\smlnj}{\textsc{SML/NJ}}
\newcommand{\mlj}{\textsc{MLj}}
\newcommand{\cml}{\textsc{CML}}
\newcommand{\pml}{\textsc{PML}}
\newcommand{\ocaml}{\textsc{OCaml}}
\newcommand{\mlkk}{\textsc{ML2000}}
\newcommand{\haskell}{\textsc{Haskell}}
\newcommand{\mltwok}{\textsc{ML2000}}
\newcommand{\scala}{\textsc{Scala}}
\newcommand{\perl}{\textsc{Perl}}
\newcommand{\scheme}{\textsc{Scheme}}
\newcommand{\unix}{\textsc{Unix}}
\newcommand{\smalltalk}{\textsc{Smalltalk}}
\newcommand{\self}{\textsc{Self}}

%
% font commands
\providecommand{\bftt}[1]{{\ttfamily\bfseries{}#1}}
\providecommand{\ittt}[1]{{\ttfamily\itshape{}#1}}
\providecommand{\kw}[1]{\bftt{#1}}
\providecommand{\nt}[1]{{\rmfamily\itshape{#1}}}
\providecommand{\term}[1]{{\sffamily{#1}}}
%
% math-mode versions
\providecommand{\mkw}[1]{\ensuremath{\text{\kw{#1}}}}
\providecommand{\mnt}[1]{\ensuremath{\text{\nt{#1}}}}
\providecommand{\mterm}[1]{\ensuremath{\text{\term{#1}}}}

% braces (in math mode)
\newcommand{\LCB}{\mkw{\{}}
\newcommand{\RCB}{\mkw{\}}}

% underscore
\newcommand{\US}{\char`\_}

%%%%%
% Some common math notation
%

% double brackets
\newcommand{\LDB}{\ensuremath{[\mskip -3mu [}}
\newcommand{\RDB}{\ensuremath{]\mskip -3mu ]}}

\newcommand{\dom}{\ensuremath{\mathrm{dom}}}
\newcommand{\rng}{\ensuremath{\mathrm{rng}}}

% sets
\newcommand{\SET}[1]{\ensuremath{\{#1\}}}
\newcommand{\Fin}{\textrm{Fin}}     % finite power set
\newcommand{\DISJOINT}[2]{\ensuremath{#1 \pitchfork #2}}
\newcommand{\finsubset}{\mathrel{\stackrel{\textrm{fin}}{\subset}}}


% finite maps
\newcommand{\finmap}{\mathrel{\stackrel{\textrm{fin}}{\rightarrow}}}
\newcommand{\MAP}[2]{\SET{#1 \mapsto #2}}
\newcommand{\EXTEND}[2]{\ensuremath{#1{\pm}#2}}
\newcommand{\EXTENDone}[3]{\EXTEND{#1}{\MAP{#2}{#3}}}
\newcommand{\SUBST}[3]{\ensuremath{#1[#2\mapsto{}#3]}}
\newcommand{\SUBSTTWO}[5]{\ensuremath{#1[#2\mapsto{}#3,#4\mapsto{}#5]}}


% timestamp
\newcount\timeHH
\newcount\timeMM
\timeHH=\time
\divide\timeHH by 60
\timeMM=\time
\count255=\timeHH
\multiply\count255 by -60 \advance\timeMM by \count255
\newcommand{\timestamp}{%
  \today{} ---
  \ifnum\timeHH<10 0\fi\number\timeHH\,:\,\ifnum\timeMM<10 0\fi\number\timeMM}



\title{Manticore Implementation Note \\ Clone Translation for Parallel Tuples}
\author{The Manticore Group}
\date{Draft of \today}

\begin{document}
\maketitle

\section{Introduction}
The idea of clone translation originates from work on the Cilk-5 language.

In this document, we describe an adaptation of clone translation that we use for
our \emph{parallel tuple} construct.
In \secref{sec:clone-translation}, we describe the general technique used by 
our compiler.
In \secref{sec:exceptions}, we describe our extension to the clone translation
to support exceptions.
In \secref{sec:avoiding-code-blowup}, we propose solutions to two code-blowup 
problems of the clone translation.
Although our presentation of the clone translation is not new, the code-blowup
problems and our solutions to them are new.

\section{Clone translation for parallel tuples}
\label{sec:clone-translation}

Our clone translation, shown in \figref{fig:ptup-clone} is as follows.  We translate each parallel
tuple into a pair of clones, called the \emph{fast clone} and the
\emph{slow clone}.  The fast clone is specialized for the common case when
the worker evaluates its own portion of the computation.  The fast
clone evaluates the computation sequentially, until either all
subcomputations have been evaluated or one or more subcomputations
have been stolen.  In the common case where none of the
subcomputations are stolen, the fast clone only incurs a small
overhead for enqueuing subcomputations and for checking for steals.

The slow clone is the same as the fast clone, except that the slow
clone supports parallel evaluation.  Because the slow clone evaluates
in parallel with the rest of the computation, some form of synchronization
is necessary, so clearly the slow clone is not as efficient as the fast 
clone.  But once the slow clone executes, all of its subcomputations execute 
as fast clones. This fact, coupled with the bound on the number of steals, 
gives us high confidence that the number of slow clones executed in the course of
a computation is insignificant, and therefore they do not play a
role in overall performance.

We show an example of our clone translation for the parallel tree-add function.
The original function, which uses parallel tuples, is given in \figref{fig:tree-prod-ptup},
and its clone translation is given in \figref{fig:tree-prod-clone}.

\subsection{Lazy promotion}
We use the technique of lazy promotion to avoid promoting all data used by the common case
(\eg{} the fast clone) by delaying promotion until the rare case (\eg{} the slow clone)
where promotion becomes necessary.
We do not handle lazy promotion explicitly in the translations shown in this document,
but below we describe, through an example, our changes to support lazy promotion
in our translations.

For example, in \figref{fig:tree-prod-clone}, this data allocated for the parallel tuple 
consists of just the ref cells \texttt{resume}, \texttt{xL} and \texttt{xR}.
In lazy promotion, we promote each ref cell at each of its \emph{use} sites (\eg{}, wherever
the ref cell is read from or written to).
Notice that only the slow clone ever needs to do any promotions.

\begin{figure}
  \input{code/tr-prod-ptup}
  \caption{Tree product using parallel tuples.}
  \label{fig:tree-prod-ptup}
\end{figure}

\begin{figure}
  \input{code/tr-prod-ptup-clone}
  \caption{Tree product after the clone translation.}
  \label{fig:tree-prod-clone}
\end{figure}

\begin{figure}
\begin{center}
\begin{code}\cdmath{}
\LDB{} \kw{(|} e\(\sb{1}\), \ldots{}, e\(\sb{n}\) \kw{|)} \RDB{}
\end{code}%
$\Longrightarrow$ \\
\begin{code}\cdmath{}
\kw{callcc}(\kw{fn} k \kw{=>}
  \kw{let} \kw{val} nDone \kw{=} \kw{ref} 1
      \kw{val} (x\(\sb{1}\), \ldots{}, x\(\sb{n}\)) \kw{=} (\kw{ref} NONE, \ldots{}, \kw{ref} NONE)
      \kw{fun} join i \kw{=} 
          \kw{if} atomicFetchAndAdd (nDone, i) \kw{=} $n$ - i + 1 
          \kw{then} \kw{throw} k(valOf (\kw{!}x\(\sb{1}\)), \ldots{}, valOf (\kw{!}x\(\sb{n}\)))
          \kw{else} steal()
      \kw{fun} waitOn x\(\sb{i}\) \kw{=} \kw{if} (\kw{!}x\(\sb{i}\)) \kw{=} NONE \kw{then} waitOn x\(\sb{i}\) \kw{else} ()
      \kw{fun} f\(\sb{n}\) () \kw{=} \LDB{} e\(\sb{n}\) \RDB{}                        
      \kw{val} w\(\sb{n}\) \kw{=} pushNew (\kw{fn} () \kw{=>} (
                 x\(\sb{n}\) \kw{:=} SOME (f\(\sb{n}\)() 
                    \kw{handle} e \kw{=>} (waitOn x\(\sb{n-1}\); \ldots{} waitOn x\(\sb{1}\); 
                                 \kw{raise} e))); 
                 join 1))
           \ldots{} 
      \kw{fun} f\(\sb{2}\) () \kw{=} \LDB{} e\(\sb{2}\) \RDB{} \kw{handle} e \kw{=>} (cancel w\(\sb{3}\); \ldots{} cancel w\(\sb{n}\);
                                     \kw{raise} e)
      \kw{val} w\(\sb{2}\) \kw{=} pushNew (\kw{fn} () \kw{=>} (
                 x\(\sb{2}\) \kw{:=} SOME (f\(\sb{2}\)() \kw{handle} e \kw{=>} (waitOn x\(\sb{1}\); 
                                               \kw{raise} e)); 
                 join 1)) 
      \kw{val} v\(\sb{1}\) \kw{=} \LDB{} e\(\sb{1}\) \RDB{} 
               \kw{handle} e \kw{=>} (cancel w\(\sb{2}\); \ldots{} cancel w\(\sb{n}\); \kw{raise} e)
  \kw{in}
      \kw{if} popNew() \kw{<>} NONE \kw{then}
          \ldots{}
          \kw{let} \kw{val} v\(\sb{n-1}\) \kw{=} f\(\sb{n-1}\)()
          \kw{in}
            \kw{if} popNew() \kw{<>} NONE \kw{then}
                (v\(\sb{1}\), \ldots{}, f\(\sb{n}\)())
            \kw{else}
                (x\(\sb{1}\) \kw{:=} SOME v\(\sb{1}\); \ldots{} x\(\sb{n-1}\) \kw{:=} SOME v\(\sb{n-1}\); 
                 join($n$ - 1))
          \kw{end}
          \ldots{}
      \kw{else}
          (x\(\sb{1}\) \kw{:=} SOME v\(\sb{1}\); join 1)
  \kw{end})
\end{code}
\end{center}
 \caption{The full clone translation for parallel tuples.}
 \label{fig:ptup-clone}
\end{figure}%

\section{Supporting exceptions}
\label{sec:exceptions}
The translation can be extended to deliver exceptions in left-to-right order
as prescribed by our sequential semantics. The interesting case
occurs when an exception is raised while evaluating the slow clone 
corresponding to the $i^{th}$ element of the tuple. Since elements in
positions to the left of $i$ might also raise exceptions, we must wait
to raise $i$'s exception until each of these other elements have finished
evaluating. By not doing so, we would allow multiple exceptions to be
raised out of order, which is clearly incorrect behavior. For efficiency, 
we cancel the tasks corresponding to elements to the right of $i$, including
those that have themselves raised exceptions and are waiting for
$i$ to complete.

\section{Avoiding code blowup}
\label{sec:avoiding-code-blowup}
The translation for parallel tuples avoids code blowup in two ways:
\begin{enumerate}
\item We lift each expression (except the first one) into its own function,
which can be safely shared by the fast and slow clone.
\item For large parallel tuples, \eg{}, those having more than 4 elements, we 
break the construction into two phases, one that builds an
intermediate nested tuple in parallel and one that builds a flattened
copy of this nested tuple. We limit the size of each nested tuple 
to a small constant. This strategy prevents a quadratic blowup introduced
by cascading memory writes performed during each transition 
from the fast clone to the slow clone (\eg{}, in our \cd{trProd} example above
one such update is the update to \cd{xL}). In general, such a
transition occurs just after evaluating 
the $i^{th}$ element of the tuple, and the code generated for the transition
must perform $i$ memory writes to record the elements of the tuple computed so 
far. Since there are as many of these transitions as there are elements in the
tuple, the number of memory writes is quadratic in the size of the parallel 
tuple given as input to the flat clone translation.
\end{enumerate}

\end{document}
